\documentclass[11pt]{article}
\input{headers7}

\usepackage{fancyhdr}   
\pagestyle{fancy}      
\lhead{MA453 Spring 2018 - Homework 7}               
\rhead{Harris Christiansen (christih@purdue.edu)}

\usepackage{mathrsfs}
\usepackage[strict]{changepage}  
\newcommand{\nextoddpage}{\checkoddpage\ifoddpage{\ \newpage\ \newpage}\else{\ \newpage}\fi}

\begin{document}

\title{Homework 7}
\date{p131 D.1 and D.2, p143 C.1, C.3, C.4, and C.8}
\maketitle

\thispagestyle{fancy}  
\pagestyle{fancy}      

\begin{enumerate}

%%% Problem p131 D.1
\item {\bfseries p131 D.1} Let $G$ be a finite group, and let $H$ and $K$ be subgroups of $G$. Prove the following:
	
	Suppose $H \subseteq K$ (therefore $H$ is a subgroup of $K$). Then $(G:H)=(G:K)(K:H)$.
  
	{\bfseries Solution.}
	
	\begin{proof}
	By definition $(G:H) =$ order of $G$ / order of $H$.
	
	Thus, $(G:K)(K:H) =$ (order of $G$ / order of $K$) * (order of $K$ / order of $H$), which can be simplified to order of $G$ / order of $H$.
	\end{proof}


%%% Problem p131 D.2
\item {\bfseries p131 D.2} Let $G$ be a finite group, and let $H$ and $K$ be subgroups of $G$. Prove the following:
	
	The order of $H \cap K$ is a common divisor of the order of $H$ and the order of $K$.
  
	{\bfseries Solution.}
	
	\begin{proof}
	Let $h$ be the order of $H$, $k$ be the order of $K$, and $i$ be the order of $H \cap K$.
	
	By Lagrange's theorem, since $H \cap K$ is a subgroup of $H$, it follows that $i$ divides $h$. Additionally, since $H \cap K$ is a subgroup of $K$, it follows that $i$ divides $k$.
	
	Thus the order of $H \cap K$ is a common divisor of the order of $H$ and the order of $K$.
	\end{proof}

%%% Problem p143 C.1
\item {\bfseries p143 C.1} Let $G$, $H$, and $K$ be groups. Prove the following:
 
	If $f : G \rightarrow H$ and $g : H \rightarrow K$ are homomorphisms, then their composite $g \circ f : G \rightarrow K$ is a homomorphism.
  
	{\bfseries Solution.}
	
	\begin{proof}
	Let $a, b \in G$. The composite 	$(g \circ f)(a*b) = g(f(a*b)) = g(f(a) * f(b)) = g(f(a)) * g(f(b)) = (g \circ f)(a) * (g \circ f)(b)$. Thus, the composite is a homomorphism.
	\end{proof}


%%% Problem p143 C.3
\item {\bfseries p143 C.3} Let $G$, $H$, and $K$ be groups. Prove the following:
 
	If $f : G \rightarrow H$ is a homomorphism and K is any subgroup of $G$, then $f(K) = \{f(x) : x \in K\}$ is a subgroup of $H$.
  
	{\bfseries Solution.}
	
	\begin{proof}
	To prove $f(K) = \{f(x) : x \in K\}$ is a subgroup of $H$, we must show:
	\begin{itemize}
		\item The subgroup is closed, as it is created from a closed group.
		\item The subgroup is closed under inversion.
		\item For $a, b \in K$, there must exist $a*b \in K$, which is true since $K$ is a subgroup of $G$. Additionally, since $f : G \rightarrow H$ is a homomorphism and K is any subgroup of G, we know $f(K)$ is closed.
		\item For $a \in K$, there must exist $a^{-1} \in K$ because K is a subgroup of G. Additionally, since $f$ is a homomorphism, we know $f(K)$ is closed under inversion.
	\end{itemize}
	Thus, $f(K)$ is a subgroup of $H$.
	\end{proof}


%%% Problem p143 C.4
\item {\bfseries p143 C.4} Let $G$, $H$, and $K$ be groups. Prove the following:
 
	If $f : G \rightarrow H$ is a homomorphism and J is any subgroup of $H$, then
	$$f^{-1}(J) = \{x \in G : f(x) \in J\}$$
	is a subgroup of $G$. Furthermore, ker $f \subseteq f^{-1}(J)$.
  
	{\bfseries Solution.}
	
	\begin{proof}
	We must show the same criteria as the previous problem.
	Because $f$ is a homomorphism and $J$ is a subgroup of $H$, we know these criteria are satisfied.
	Thus, $f^{-1}(J)$ is a subgroup of $G$.
	\end{proof}


%%% Problem p143 C.8
\item {\bfseries p143 C.8} Let $G$, $H$, and $K$ be groups. Prove the following:
 
	The function $f : G \rightarrow G$ defined by $f(x) = x^2$ is a homomorphism iff $G$ is abelian.
  
	{\bfseries Solution.}
	
	\begin{proof}
	Prove both directions:
	
	Suppose $f$ is a homomorphism. Thus $f(x*y) = f(x)*f(y)$ for $x, y \in G$. Using the definition of $f$ we get $f(xy) = (xy)^2 = xyxy$. Because $f$ is a homomorphism, $f(xy) = f(x)f(y) = x^2y^2$, thus $G$ must be abelian.
	
	Suppose $G$ is abelian, meaning $xy = yx$. We can show $f(xy) = (xy)^2 = xyxy = xxyy = x^2y^2 = f(x)f(y)$. Thus, because $G$ is abelian, $f$ must be a homomorphism.
	\end{proof}

\end{enumerate}

\end{document}
